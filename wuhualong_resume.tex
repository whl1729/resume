
\documentclass[letterpaper,11pt]{article}

\usepackage{latexsym}
\usepackage[empty]{fullpage}
\usepackage{titlesec}
\usepackage{marvosym}
\usepackage[usenames,dvipsnames]{color}
\usepackage{verbatim}
\usepackage{enumitem}
\usepackage[hidelinks]{hyperref}
\usepackage{fancyhdr}
\usepackage[english]{babel}

\pagestyle{fancy}
\fancyhf{} % clear all header and footer fields
\fancyfoot{}
\renewcommand{\headrulewidth}{0pt}
\renewcommand{\footrulewidth}{0pt}

% Adjust margins
\addtolength{\oddsidemargin}{-0.5in}
\addtolength{\evensidemargin}{-0.5in}
\addtolength{\textwidth}{1in}
\addtolength{\topmargin}{-.5in}
\addtolength{\textheight}{1.0in}

\urlstyle{same}

\raggedbottom
\raggedright
\setlength{\tabcolsep}{0in}

% Sections formatting
\titleformat{\section}{
  \vspace{-4pt}\scshape\raggedright\large
}{}{0em}{}[\color{black}\titlerule \vspace{-5pt}]

%-------------------------
% Custom commands
\newcommand{\resumeItem}[2]{
  \item\small{
    \textbf{#1}{: #2 \vspace{-2pt}}
  }
}

\newcommand{\resumeSubheading}[4]{
  \vspace{-1pt}\item
    \begin{tabular*}{0.97\textwidth}[t]{l@{\extracolsep{\fill}}r}
      \textbf{#1} & #2 \\
      \textit{\small#3} & \textit{\small #4} \\
    \end{tabular*}\vspace{-5pt}
}

\newcommand{\resumeSubItem}[2]{\resumeItem{#1}{#2}\vspace{-4pt}}

\renewcommand{\labelitemii}{$\circ$}

\newcommand{\resumeSubHeadingListStart}{\begin{itemize}[leftmargin=*]}
\newcommand{\resumeSubHeadingListEnd}{\end{itemize}}
\newcommand{\resumeItemListStart}{\begin{itemize}}
\newcommand{\resumeItemListEnd}{\end{itemize}\vspace{-5pt}}

\usepackage{
  array,
  booktabs,
  hyperref,
  color,
  latexsym,
  verbatim,
  url,
  ulem,
  xeCJK, % replace with CJK for sharelatex.com
  multirow,
  enumitem,
  fancyhdr,
  tikz,
  calc, % fix hbox too wide for heading, enable caculate
  titlesec
}



\begin{document}

%----------HEADING-----------------
\begin{center}
  \small \textbf{\href{http://www.cnblogs.com/wuhualong/}{\huge 伍华龙}} \\  
  136-9167-8556 $\vert$
  \href{mailto:2393361669@qq.com}{2393361669@qq.com} $\vert$
  深圳市南山区 \\
  \href{https://github.com/whl1729}{\color{blue}{github.com/whl1729}} $\vert$
  \href{http://www.cnblogs.com/wuhualong/}{\color{blue}{cnblogs.com/wuhualong/}} \\
\end{center}

%-----------EDUCATION-----------------
\section{教育经历}
  \resumeSubHeadingListStart
    \resumeSubheading
      {中山大学}{广州}
      {电子与信息工程学院,通信工程学士}{2012.09 - 2016.06}
  \resumeSubHeadingListEnd

%-----------SKILLS-----------------
\section{专业技能}
  \begin{itemize}[leftmargin=*]
    \item \textbf{编程语言}: 熟悉C/C++和shell,了解汇编、C\#和Python
    \item \textbf{开发环境}: 熟悉vim、Markdown、Source Insight、Git/SVN、gdb、gcc、ld、make等开发工具,熟悉Linux常用操作,了解Jenkins
    \item \textbf{专业基础}: 熟悉数据结构与算法、操作系统、计算机组成原理、计算机网络,了解设计模式
  \end{itemize}

%-----------EXPERIENCE-----------------
\section{工作经历}
  \resumeSubHeadingListStart

    \resumeSubheading
      {华为技术有限公司}{深圳坂田}
      {嵌入式软件开发工程师}{2016.07 - 2018.09}
      \resumeItemListStart
        \resumeItem{品质宽带}
          {华为在业界推出的智能家宽运维解决方案,部门年度重点项目之一。
            \begin{itemize}
                \item 使用C语言实现单板数据采集上报功能。其中采用YANG模型来组织数据,并对数据进行GPB格式编码,然后通过GRPC协议建立单板与NCE(网络云化引擎)连接。
                \item 使用makefile和shell脚本搭建编译工程,使用Jenkins搭建CI工程,每天定时编译、构建、冒烟,初步保证版本发布的基本功能。
            \end{itemize}
          }
        \resumeItem{配置管理}
          {包括CI部署、代码度量、版本管理和发布等。
            \begin{itemize}
                \item 使用PC-lint、SourceMonitor、CodeDex等工具搭建门禁工程,开发者合入代码时会自动触发门禁构建,门禁检查失败会禁止合入,初步保证代码合入质量。每天定期冒烟测试,进一步确保版本满足基本业务功能。
                \item 编写shell脚本,一键式完成版本号修改、版本提交、环境升级等操作,版本发布效率提高5倍以上。
                \item 在公司内部平台发表多篇高质量wiki,指导新CMO快速入门以及新员工解决常见开发环境问题,得到同事和主管的一致认可。
            \end{itemize}
          }
        \resumeItem{单板小APP}
          { 用于解决单板软件出现故障后无法正常启动的问题,能够提高系统健壮性和团队开发效率。
            \begin{itemize}
                \item 作为开发人员,使用C语言裁剪和修改本领域原有代码,并新增调试手段,提高问题定位效率。
                \item 作为特性owner,协调、推动各领域进行开发、联调和问题定位,在时间紧迫的情况下按时完成交付,转测试后零缺陷。
            \end{itemize}
          }
        \resumeItem{NAT}
          {使用C语言在局端设备上实现NAT功能,使得同一局域网内的终端设备可以共用一个ip,从而解决ip资源短缺的问题。负责代码开发及联调,代码量约5K。}
      \resumeItemListEnd

    \resumeSubheading
      {华为技术有限公司}{深圳坂田}
      {软件开发实习生}{2015.07 - 2015.12}
      \resumeItemListStart
        \resumeItem{寄存器代码生成工具}
          {使用C\#语言读取Excel表格和Perl文件中的寄存器信息,同时生成xml文件和C语言代码。该工具界面简洁友好,方便开发人员查询寄存器信息,并自动生成C语言代码,有效提升部门开发效率。}
      \resumeItemListEnd

  \resumeSubHeadingListEnd

%-----------PROJECTS-----------------
\section{项目经历}
  \resumeSubHeadingListStart
    \resumeSubItem{\href{https://github.com/whl1729/ucore_os_lab}{ucore os lab}}
      {清华大学ucore操作系统开源项目。基于MIT xv6的设计,实现一个简单的操作系统,支持虚存管理、进程管理、处理器调度、同步互斥、进程间通信、文件系统等主要内核功能。}
    \resumeSubItem{俄罗斯方块}
      {大三微机原理实验课期末设计。使用汇编语言实现经典游戏“俄罗斯方块”,并成功在实验箱上运行,最终本门实验课我获得全班最高分。}
  \resumeSubHeadingListEnd

%-----------Awards-----------------
\section{获奖情况}
  \resumeSubHeadingListStart
  \small
    \item{2012-2013、2014-2015学年均获得\textbf{中山大学本科生优秀奖学金}\textit{三等奖}}
  \resumeSubHeadingListEnd
%-------------------------------------------
\end{document}
